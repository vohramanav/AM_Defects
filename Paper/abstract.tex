\section*{Abstract}

An efficient approach referred to as the PCAS method for surrogate modeling in the presence of
high-dimensional input and output is proposed. Specifically, a large
collection of input variables is mapped to an output of interest which is a field quantity as opposed
to a scalar.
Computational efficiency is accomplished by first identifying principal components~(PC)
or directions in  the output field data. A set of representative features corresponding to the important directions
of the high-dimensional output is 
determined. A map from the set of inputs in the physical space to each feature is considered,
and the active subspace~(AS) methodology is used to capture their relationship in a low-dimensional subspace in the input 
domain. The map for each feature of the output is then approximated by a surrogate model in the active subspace. Thus, the PCAS
method aims to accomplish dimension reduction in the input as well as the output by first mapping the set of inputs
in the physical space to fewer variables in the active subspace~(active variables), and linking these active variables
to a set of representative features
that approximate the field of interest. 
The surrogate model is constructed in an iterative manner to reduce computational burden
associated with training using the expensive physics model. 
The proposed framework is implemented to build a surrogate model for the purpose of reliability analysis
with respect to residual stress in an additively
manufactured component. The surrogate model is further exploited for detecting 
stress hotspots in the component, and for global sensitivity analysis to determine the impact of manufacturing process variables and
material properties on the development of residual stress. Our findings based on the considered application
are indicative of enormous potential for computational gains in such analyses. 

\bigskip

\noindent \textbf{Keywords}: Principal components, active subspace, surrogate model, dimension reduction,
residual stress, additive manufacturing
