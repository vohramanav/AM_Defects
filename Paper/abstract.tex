\section*{Abstract}

An efficient approach referred to as the PCAS method for surrogate modeling in high input and output dimensions is 
proposed. Specifically, a large
collection of input variables is mapped to a field of interest as opposed to a scalar model output or a system response.
Computational efficiency is accomplished by first identifying principal components~(PC)
or directions in  the output field data. A set of representative features corresponding to the important directions is 
determined. A map from the set of inputs in the physical space to each feature is considered,
and the active subspace~(AS) methodology is used to capture their relationship in a low-dimensional subspace in the input 
domain. The map for each feature is hence approximated by a surrogate in the active subspace. Thus, the PCAS
method aims at \textit{compounded} dimension reduction wherein the inputs are mapped to a set of representative features
that approximate the field of interest. The resulting map is referred to as the \textit{composite surrogate}.
The proposed framework is implemented to build a surrogate for the purpose of reliability analysis
based on residual stress in an additively
manufactured component using the electron beam melting process. The surrogate is further exploited for detecting 
stress hotspots in the part, and global sensitivity analysis to determine the impact of process variables and
alloy material properties on the development of residual stress. Our findings based on the considered application
are indicative of enormous potential for computational gains for such analyses. 

\bigskip

\noindent \textbf{Keywords}: Principal components, active subspace, surrogate, dimension reduction,
residual stress, additive manufacturing
