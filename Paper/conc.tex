\section{Summary and Discussion}
\label{sec:conc}

In this paper, we have proposed an efficient approach, namely the PCAS method for constructing a surrogate model that 
maps a high-dimensional input to a high-dimensional output. The high-dimensional output is considered as a field quantity,
estimated at discrete points on a mesh used for numerical simulations. Computational efficiency is
accomplished by means of dimension reduction in the output space as well as the input space.
We begin by determining the optimal number of
components required to reasonably approximate the field using an iterative PCA approach~(Algorithm~\ref{alg:pca}).
The optimal set of components or important directions are hence used to determine a set of features that
are representative of the field of interest. The number of features is the same as the number of optimal components.
Each feature is regarded as a function of the set of input variables. Variability in a given feature due to the variability
in the inputs is captured in a low-dimensional subspace using the active subspace methodology discussed
in~\ref{sub:as}. The PCAS method thus enables compounded dimension reduction since it reduces the dimensionality
of the map from a set of inputs to key features in the output. Computational efficiency is enhanced by constructing 
a surrogate in the active subspace computed for each feature. Therefore, the overall map from the input space
to the output field of interest comprises an array of individual surrogates, and is therefore regarded as the
composite surrogate. It is expected that the computational efficiency is accomplished with a trade-off in accuracy.
Therefore, it is critical to perform a robust verification and validation of the resulting surrogate model as discussed
in~\ref{subsub:vnv}. 

The proposed methodology is demonstrated using an engineering application pertaining to reliability analysis of
an additively manufactured part. Specifically, we focused our efforts on predicting the development of residual
stress in a part at the end of an electron beam melting process using a finite element model in Abaqus.
Due to the limited availability of computational
resources, we considered a single pass of the beam in this work. Nevertheless, it was found to be sufficient for
illustrating the potential of the PCAS method. The von Mises stress field in a 2-dimensional non-uniform mesh
in a cross-section of the AM part was considered as residual stress and the field of interest. It was found that
7 features were able to approximate the stress field using the iterative PCA approach. The set of inputs
comprising the process control parameters, mechanical and thermal properties of the alloy (used to manufacture the
AM part) were mapped to each of these 7 features. A 1-2 dimensional active subspace was shown to reasonably
capture the dependence of each feature on the inputs thereby indicating enormous scope for computational gains.
The surrogate was shown to be remarkably accurate by estimating the relative L-2 norm of the discrepancy
between the model output and the field reconstructed using the composite surrogate. Specifically, on average, the
surrogate achieved an accuracy of about 4$\%$ and 7$\%$ in the verification and validation tests respectively.

The surrogate was used to detect hotspots in the AM part~(Section~\ref{subsub:hotspot}), and global sensitivity
analysis of the process variables, mechanical, and thermal properties of the alloy~(Section~\ref{sub:gsa}). 
The hotspots were observed to be in the proximity of the applied heat source, i.e. closer to the surface of the
AM part, thereby indicating that the residual stress is dominated by the presence of large temperature gradients.
The GSA results 
indicate that the residual stress is relatively more sensitive towards the material properties, although the sensitivity
towards the process variables is also found to be significant due to their interactions with the material properties,
accounted for in the total-effect index. Finally, the composite surrogate was exploited to numerically estimate the
probability of failure using a million samples in the input domain for the purpose of reliability analysis of the AM part. 
The part was considered to have failed if the von Mises stress at the global hotspot (point P) exceeded a limiting 
value of 900~MPa. The probability of failure was estimated to be 0.177. 

It must be highlighted that the aforementioned
analyses such as hotspot detection, GSA, and reliability prediction are typically computationally intensive and
infeasible in additive manufacturing.
 The composite surrogate constructed using the PCAS method makes them computationally
feasible while ensuring a reasonable amount of accuracy for the present application.
 However, there are limitations that should be considered
when applying the proposed framework. First, dimension reduction in the output space is conditioned upon the
existence of a structure in the data that could be captured by a relatively small number of principal components or
directions. Second, a low-dimensional active subspace can be used to map the set of inputs to the quantity of
 interest~(QoI).
To satisfy this requirement, the gradient of the QoI with respect to each input should be estimated
with reasonable accuracy. For the application presented in this work, we have used a regression-based approach for 
estimating the gradients that resulted in a reasonably accurate surrogate for each feature of the output field of interest.
However, depending upon the relationship between the QoI and the set of inputs, a relatively more accurate approach
such as those involving perturbation techniques~(e.g.~automatic differentiation~\cite{Kiparissides:2009}, adjoint 
methods~\cite{Borzi:2011, Alexanderian:2017}) may be required. Additionally, the active subspace methodology is
not suitable in situations where the gradient is not continuous in the entire domain of the inputs.

To sum up, the proposed methodology was successfully demonstrated for a reasonably challenging practical
application involving reliability analysis based on residual stress in an AM part, manufactured using the EBM process. 
Enormous computational gains leading to dimension reduction by two orders of magnitude was accomplished. Therefore,
the proposed framework seems like a powerful tool for surrogate building in applications involving large input and
output dimensions. Our future efforts will focus on further development of the proposed framework to enhance its
applicability as a prognosis tool for process control and optimization as well as defect characterization and minimization in 
additive manufacturing.

























