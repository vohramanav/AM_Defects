\section{Introduction}
\label{sec:intro}

%1. Motivate the need for a fast surrogate modeling approach: expensive training, large input and output dimensions
%(include applications)
%2. Briefly discuss our approach
%3. Briefly discuss the application: motivate residual stress analysis, reliability analysis is expensive
%4. Key contributions of the paper
%5. Outline

%A surrogate model also referred to as an emulator or a response surface,
%is a widely used and powerful tool that enables a range of applications pertaining to 
%computational science. As the name suggests, a surrogate aims to capture the nature of dependence of a system
%response or a model output on its inputs and parameters. It is typically informed by the underlying physics in the 
%case of a physics-based model or by data in the case of supervised learning~(e.g. neural networks~\cite{Hagan:1996}).
%A reliable surrogate could thus
%be used in lieu of the model for making predictions in a regime where it is validated. 

A surrogate model can offer a significant computational advantage in situations involving intensive
simulations, especially for applications
involving a large number of model evaluations such as uncertainty propagation, sensitivity analysis, parameter
estimation, and optimization. However, constructing a reasonably accurate surrogate itself can be computationally demanding.
For instance, estimation of coefficients of a polynomial chaos expansion (PCE)~\cite{Xiu:2002,Ghanem:1991},
a commonly used surrogate
in scientific applications, suffers from the so-called `curse of dimensionality'. Although remarkable progress has been 
made towards efficient computation of the PC coefficients~(e.g. sparse 
grids~\cite{Gerstner:1998,Ganapathysubramanian:2007} and basis adaptive 
methods~\cite{Blatman:2011,Conrad:2013,Winokur:2013}), it remains computationally challenging for high-dimensional 
applications especially if the output quantity of interest~(QoI) is a field as opposed 
to a scalar quantity. Similarly, in the case of Gaussian 
Process~\cite{Rasmussen:2004} surrogate modeling, computing the inverse of the correlation matrix
becomes challenging in large dimensions. 

In this paper, we develop a novel surrogate modeling approach that focuses on combining the dimensionality reduction
in the \textit{output}
space wherein the observation is a field quantity (as opposed to scalar) with dimensionality reduction in the \textit{input} 
space. More specifically, we first exploit principal component analysis to extract key features in the output. Then, we 
discover a 
low-dimensional structure in the relationship between representative features of the output and the set of inputs using
the active
subspace methodology~\cite{Constantine:2015}. The proposed methodology is
referred to as the PCAS method in this work as it combines principal components~(PC) with active subspaces~(AS).
The framework is implemented to
perform a reliability analysis of an additively manufactured~(AM) part by assessing the development of residual stress
at the end of a single pass of a laser scan in electron beam melting (EBM). 

Residual stress develops during the manufacturing process due to the presence of steep thermal gradients as
well as physical constraints in the part which adversely affect its mechanical properties, geometry, and 
shape~\cite{Withers:2001,Mercelis:2006,Hofmann:2014}. 
In fact, residual stress in addition to porosity is one of the main reasons for 
part failure~\cite{Kim:2018}. The presence of residual stress in an AM part has significantly 
inhibited rapid certification as well as standardization of the certification process
due to post processing involving machining and heat treatment~\cite{Shiomi:2004}.
Several recent investigations~\cite{Vastola:2016,Hodge:2016,Williams:2018}
have focused on developing thermo-mechanical
models to better understand the development of residual stress and optimize the microstructure as well as
the process control parameters accordingly. However, since the simulations are intensive and models require a 
large amount of calibration data, the progress has so far been limited by the availability
of computational and experimental resources. Through this study, we aim to demonstrate an effective strategy
based on surrogate modeling that could accelerate material selection, microstructure design, and
process control and optimization for controlling the evolution residual stress during additive manufacturing. 

Residual stress in an AM part is computed using a finite element thermo-mechanical model in Abaqus in this work. 
More specifically, the finite element model~(FEM) includes a thermal model that simulates the thermal response of the
part. Part thermal response is then used as an input to a mechanical model that predicts residual stress at the end
of a cooling phase. For a given set of process conditions and material properties, the thermal model requires
approximately 20 minutes to generate the temperature field and the mechanical model takes approximately 10
minutes to estimate the residual stress in the part. Therefore, one realization of the output field of interest
using the FEM requires approximately 20 minutes. In order to perform reliability analysis using
sampling techniques, $\mathcal{O}(10^4--10^5)$ realizations are typically required for reasonable accuracy.  
Therefore, it is not practical
to rely on the FEM for this purpose. Additionally, conventional approaches for surrogate
modeling would require a large amount of computational resources for the purpose of training as discussed earlier.
 A random field approximation is a possibility for
output dimension reduction. However, such an approximation could potentially introduce large errors in the  
representation of the field. Instead, we aim to exploit the structure in the data from an FEM by identifying important 
directions or principal components in the field. This approach allows us to select an optimal number of 
features required to re-construct the field with reasonable accuracy and computational effort. 

Key contributions of this paper can be summarized as follows: (1) A computationally efficient approach is developed
for constructing an efficient surrogate for problems where a large set of inputs are mapped to a large-dimensional
field data. (2) A finite element model is developed to simulate residual stress in an additively manufactured part
at the end of a single scan of the laser beam in an EBM process. (3) The surrogate is used to perform a global
sensitivity analysis~(GSA) to assess relative importance of the material properties and the process control parameters
in the context of residual stress. (4) Finally, the surrogate is used for the purpose of reliability analysis by estimating the
probability of failure using \textit{hotspot detection} in the part. 

The remainder of this paper is organized as follows: Section~\ref{sec:method} outlines the proposed methodology for
constructing the surrogate including a brief background on the active subspace methodology used in this work.
Section~\ref{sec:model} details the finite element model used to generate the stress data for building the surrogate.
Section~\ref{sec:results} includes our findings and discussion pertaining to the implementation of the methodology 
for surrogate construction, GSA, and reliability analysis of the AM part. 
Finally, we summarize this study in Section~\ref{sec:conc}. 



