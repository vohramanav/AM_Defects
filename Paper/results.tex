\section{Results}
\label{sec:results}

In this section, we provide relevant details pertaining to the construction of the surrogate for the field of interest, i.e. the
residual stress in the AM part cross-section using the PCAS method in~\ref{sub:surr}. The surrogate is used to map
the process control parameters and the material properties to the residual stress field. The computational efficiency
enabled by the surrogate is exploited to perform a global sensitivity analysis of the inputs in~\ref{sub:gsa}. Finally, the
surrogate is used for reliability prediction for the AM part by estimating the probability of failure based on residual stress
in~\ref{sub:reliability}.

\subsection{Surrogate Model}
\label{sub:surr}

A surrogate model is constructed for the residual stress field at the cross-section of the part 
(x-z plane in Figure~\ref{fig:PartwMesh}) passing through its center. We will refer to this plane
as x$^c$-z$^c$ in the remainder of this paper. The surrogate maps three sets of
parameters, namely, the process control parameters~($\bm{\theta_p}$), mechanical properties~($\bm{\theta_m}$),
and thermal properties~($\bm{\theta_t}$) to the stress field. The set of process control parameters includes
the beam power~($P$), scan speed~($v$), and the pre-heat temperature~($T_0$). Mechanical properties
include the yield strength~($Y$), the elastic modulus~($E$), and the bulk density~($\rho$). Thermal 
properties include specific heat~($C_p$) and bulk thermal conductivity~($\kappa$). Note that $C_p$ and $\kappa$
are considered to be functions of the local temperature, $T$. Specifically, a polynomial of degree 2 was fit
to a set of data pertaining to the variation of $C_p$ and $\kappa$ with temperature (20~K--1655~K), 
provided in~\cite{Fu:2014} as shown in Figure~\ref{fig:Cp_kappa}.
%
\begin{figure}[htbp]
\begin{center}
\includegraphics[width=0.42\textwidth]{./Figures/cp_fit}
\includegraphics[width=0.42\textwidth]{./Figures/kappa_fit}
\end{center}
\caption{A second degree polynomial fit to specific heat~($C_p$), and thermal conductivity~($\kappa$) data
for a temperature range, [20,1655](K). Note that the data provided in~\cite{Fu:2014} was used to determine
the coefficients of the regression fit.}
\label{fig:Cp_kappa}
\end{figure}
%
Hence, a total of 12 parameters~($\bm{\theta}$) are mapped to the stress field including coefficients of the polynomial fits
corresponding to $C_p$ and $\kappa$. A uniform prior: $[0.9\bm{\theta}^\ast, 1.1\bm{\theta}^\ast]$, 
where $\bm{\theta}^\ast$ denotes a vector of nominal values,
was considered for each parameter. Nominal values of the mechanical properties: $Y$, $E$, and $\rho$ are provided
in Table~\ref{tab:matProp}. Nominal values of the process control parameters and temperature coefficients for
the thermal properties are provided in Table~\ref{tab:remain}.
%
\begin{table}[htbp]
\centering
\caption{EBM process control parameters and temperature coefficients for $C_p$~($C_i$'s) and $\kappa$~($D_i$'s).}
\label{tab:remain}
\vspace{1mm}
\begin{tabular}{ ll }
\toprule
Scan Speed, $v$~(mm/s) & 500 \\
Beam Power, $P$~(W) & 160 \\
Pre-heat Temperature, $T_0$~(C) & 650 \\
Specific heat, $C_p$ = $C_0+C_1T+C_2T^2$~(J/kg/K) & 540~($C_0$),0.43~($C_1$),$-3.2\times 10^{-5}$~($C_2$) \\
Thermal Conductivity, $\kappa$ = $D_0+D_1T+D_2T^2$~(W/m/K) & 7.2~($D_0$),0.011~($D_1$),$1.4\times 10^{-6}$~($D_2$) \\
\bottomrule
\end{tabular}
\end{table}
%%%

Residual stress was computed at the x$^c$-z$^c$ plane for 20 pseudorandom samples in the 12-dimensional
input domain. Stress data was simulated on a 2-dimensional non-uniform grid comprising 32 points along the
length~(x$^c$) and 14 points along the height~(z$^c$) as highlighted in Figure~\ref{fig:RS_comp}~(left).
As mentioned earlier in Section~\ref{sec:model}, a 
finer mesh is used near the part surface since sharp thermal gradients lead to a larger amount of stress
in this region as shown in Figures~\ref{fig:subSmises} and~\ref{fig:RS_comp}.  Following the flow diagram
in Figure~\ref{fig:fd}, the first step involves a principal component analysis on the field data. For this purpose,
the iterative PCA  algorithm~(Algorithm~\ref{alg:pca}) was used. 

In Figure~\ref{fig:pca}, we plot the reconstruction error, $\varepsilon_\mathcal{R}^\infty$ against the number of
principal components, $K$. 
%
\begin{figure}[htbp]
\begin{center}
\includegraphics[width=0.42\textwidth]{./Figures/error_PCA}
\end{center}
\caption{A plot of the reconstruction error, $\varepsilon_\mathcal{R}^\infty$ as a function of the
number of principal components as obtained using the iterative PCA approach in Algorithm~\ref{alg:pca}.}
\label{fig:pca}
\end{figure}
%
As expected, $\varepsilon_\mathcal{R}^\infty$ is observed to mostly decrease with the number of components. 
A monotonic behavior is not expected since the components only capture partial information in the data. It 
appears that most of the information is captured using 20 components since the error is almost 0. However,
building the surrogate for 20 features would potentially entail a large computational effort depending upon the
application. Here, we consider that $K^\ast$ = 7 components are optimal since the error plateaus as the number of
components increase from 7 to 10 indicating diminishing returns. Thus, the residual stress field is reconstructed
using a surrogate for each of these $K^\ast$ components~($\mathcal{Z}_i$'s, $i = 1,2,\ldots,K^\ast$).
The dimensionality of the output space is therefore reduced 
from $\mathbb{R}^{14\times 32=448}\rightarrow \mathbb{R}^7$. We now shift our focus on dimension reduction
in the input space.

As discussed earlier in~\ref{sub:as}, each feature can be expressed as a function of $\bm{\theta}$ in the physical
space. Note that $\bm{\theta}:\{\bm{\theta_p}\cup\bm{\theta_m}\cup\bm{\theta_t}\}$. An active subspace computation
was performed using a regression-based approach~\cite{Constantine:2015, Vohra:2019}
for estimating the gradient and the available set of 20 realizations for each $\mathcal{Z}_i$. 
Eigenvalue spectrum of the matrix, $\hat{\mathbb{C}_i}$ for each $\mathcal{Z}_i$ is shown in Figure~\ref{fig:as}.
Variability of a given $\mathcal{Z}_i$ in terms of the active variables, $\bm{\eta}$ regarded as the sufficient summary
plot~(SSP) is also included in each case.
%
\begin{figure}[htbp]
\begin{center}
\begin{subfigure}{0.35\textwidth}
\includegraphics[width=0.65\textwidth]{./Figures/eig_Zf1} 
\\
\includegraphics[width=0.65\textwidth]{./Figures/eig_Zf2} 
\\
\includegraphics[width=0.65\textwidth]{./Figures/eig_Zf3} 
\\
\includegraphics[width=0.65\textwidth]{./Figures/eig_Zf4} 
\\
\includegraphics[width=0.65\textwidth]{./Figures/eig_Zf5} 
\\
\includegraphics[width=0.65\textwidth]{./Figures/eig_Zf6} 
\\
\includegraphics[width=0.65\textwidth]{./Figures/eig_Zf7} 
\end{subfigure}
\hspace{-0.5cm}
\begin{subfigure}{0.35\textwidth}
\includegraphics[width=0.65\textwidth]{./Figures/SSP_Zf1} 
\\
\includegraphics[width=0.65\textwidth]{./Figures/SSP_Zf2} 
\\
\includegraphics[width=0.65\textwidth]{./Figures/SSP_Zf3} 
\\
\includegraphics[width=0.65\textwidth]{./Figures/SSP2D_Zf4} 
\\
\includegraphics[width=0.65\textwidth]{./Figures/SSP_Zf5} 
\\
\includegraphics[width=0.65\textwidth]{./Figures/SSP_Zf6} 
\\
\includegraphics[width=0.65\textwidth]{./Figures/SSP_Zf7} 
\end{subfigure}
\end{center}
\caption{Eigenvalue spectrum and the corresponding SSP for each $\mathcal{Z}_i$.}
\label{fig:as}
\end{figure}
%
From these plots, it is observed that in all cases except $\mathcal{Z}_4$, a 1-dimensional active
subspace captures the variability in the feature with reasonable accuracy. More specifically, the
eigenvalue spectrum exhibits a significant jump between the first and second eigenvalue. Consistent
with these observations, a straight-line fit to the realizations of the feature in the SSP is observed to be
reasonably accurate. In the case of $\mathcal{Z}_4$, $\lambda_1$ and $\lambda_2$ are comparable, and
a significant jump exists between $\lambda_2$ and $\lambda_3$. Therefore, a 2-dimensional active
subspace is considered. A polynomial regression surface fit is shown to capture the variability of
$\mathcal{Z}_4$ with reasonable accuracy. Polynomials of degrees 3 and 2 along $\eta_1$ and
$\eta_2$ respectively were used to construct the regression surface. The regression-fits in each case
serves as a surrogate for the corresponding feature, $\mathcal{Z}_i$. Therefore, a sample $\bm{\xi}_i$
corresponding to $\bm{\theta}_i$ in the physical space is propagated through each surrogate to estimate
$\mathcal{Z}_i$'s and hence, the residual stress field as shown using a flow diagram in Figure~\ref{fig:re}.
The individual surrogates for $\mathcal{Z}_i$'s thus constitute the overall surrogate model that maps the 
physical variables to the stress field. We will refer to this overall surrogate model as the \textit{composite
surrogate} in the remainder of this paper. 

Dimension reduction in the input space is therefore found to be from $\mathbb{R}^7\rightarrow\mathbb{R}^2$.
The overall dimension reduction is therefore, $\mathbb{R}^{448}\rightarrow\mathbb{R}^2$ which indicates
enormous scope for computational gains using the PCAS method, and enables reliability prediction for the
AM part in a tractable manner. 

\subsubsection{Surrogate Verification and Validation}
\label{subsub:vnv}

It is critical to \textit{verify} as well as \textit{validate} the accuracy of the resulting composite surrogate model
that maps variables in the physical space to the field of interest i.e., residual stress in the x$^c$-z$^c$ plane. 
For verification, we reconstruct the stress field at the same set of 20 samples used to generate realizations
of the important features in the data for building the low-dimensional surrogate in each case. However, for the
purpose of validation, stress fields constructed at an independent set of 10 samples is compared to the
corresponding FEM predictions. To quantify the accuracy of the surrogate during verification and validation,
we compute an averaged relative L-2 error norm of the discrepancy~($\varepsilon_d$) in the stress field, simulated using
the FEM and constructed using the surrogate model at the same set of grid points in the 
2D mesh~(see Figure~\ref{fig:RS_comp}~(left)). Mathematical expression for $\varepsilon_d$ is given as
follows:
%
\be
\varepsilon_{d} = \frac{1}{N}\sum\limits_{i=1}^{N} \frac{\|\mat{S}-\hat{\mat{S}}\|_2}{\|\mat{S}\|_2},
\label{eq:test}
\ee
%
where, $N$ denotes the number of samples and is equal to 20 and 10 in the case of verification and validation
respectively; $\mat{S}$ and $\hat{\mat{S}}$ denote the stress field simulated using the FEM and the composite
surrogate respectively. The discrepancy error, $\varepsilon_d$ was calculated to be approximately 0.04 and 
0.07 in the case of verification and validation respectively. In other words, the stress field reconstructed using the
composite surrogate model achieves an accuracy of about 7$\%$ on average. Although these error estimates
are based on a relatively small sample size, they seem reasonable considering that the validation test samples
were generated using Latin hypercube sampling (LHS) that explores the entire input domain more uniformly
as compared to Monte Carlo sampling. Therefore, the PCAS approach leads to a reasonably 
accurate surrogate coupled with enormous computational gains which makes the analyses pertaining to
the present application in this work, feasible. 

Figure~\ref{fig:RS_comp} illustrates a side-by-side comparison of stress distribution in the $x^c$-z$^c$ plane,
computed using the FEM~(left) with those generated using the composite surrogate~(right) using the same
set of input conditions. The two plots
are observed to be in close agreement with each other.  
%
\begin{figure}[htbp]
\begin{center}
\begin{subfigure}{0.15\textwidth}
\vspace{10mm}
\includegraphics[width=0.5\textwidth]{./Figures/xczc} 
\end{subfigure}
\hspace{-1.5cm}
\begin{subfigure}{0.35\textwidth}
\includegraphics[width=1.0\textwidth]{./Figures/origZ_sam13} 
\end{subfigure}
\hspace{0.25cm}
\begin{subfigure}{0.35\textwidth}
\includegraphics[width=1.0\textwidth]{./Figures/recZ_sam13} 
\end{subfigure}
\end{center}
\caption{Left: Residual stress field in the $x^c$-z$^c$ plane as generated using the Abaqus model with inputs, 
$\bm{\theta}_p$. The grid points in the 2D mesh used for finite element simulations are also highlighted. 
Right: Reconstructed stress field using the surrogate model using: $v=535.23$~mm/s, $P$=148.27~W,
$T_0$=641.03~K, $Y$=777.74~MPa, $E$=99.16~GPa, $\rho$=4187.25~kg/m$^3$, $C_0$=546.43, $C_1$=0.47,
$C_2$~=~-~3.07$\times$10$^{-5}$, $D_0$=6.84, $D_1$=0.01, $D_2$=1.47$\times$10$^{-6}$.}
\label{fig:RS_comp}
\end{figure}

\subsubsection{Hotspot Detection}
\label{subsub:hotspot}

As mentioned earlier in Section~\ref{sec:intro}, a large amount of residual stress severely impacts part
performance due to sub-optimal mechanical properties, reduced fatigue life, and geometrical inaccuracy. 
Detection of stress hotspots could thus be conceived as an important step in the manufacturing process.
Owing to the trasient nature of the process conditions, material microstructure, and part configuration,
it would not be practicable to use the FEM. The composite surrogate could instead be used for the purpose
of hotspot detection in the part. 

For the present analysis, we focus on the hotspots in the $x^c$-z$^c$ plane, and any point in the 2D mesh
where the stress exceeds a threshold is considered as the hotspot. Figure~\ref{fig:hs} illustrates the 
location of the hotspots and associated stress values in the $x^c$-z$^c$ plane, for a particular set of
input conditions. A threshold value of 640~MPa was used. 
%
\begin{figure}[htbp]
\begin{center}
\includegraphics[width=0.42\textwidth]{./Figures/loc_hotspot10}
\end{center}
\caption{Location of the hotspots in the AM part and corresponding estimates of the von Mises stress
are indicated by means of a colorbar. The location of the peak stress is also shown using a black square.}
\label{fig:hs}
\end{figure}
%
As expected, the hotspots are located near the top surface of the part that experiences sharp temperature
gradients. 

\subsection{Global Sensitivity Analysis}
\label{sub:gsa}

The composite surrogate was used for the purpose of global sensitivity analysis~(GSA). The residual stress
field in the $x^c$-z$^c$ plane was simulated for 10$^6$ pseudorandom samples, generating using LHS
in the 12-dimensional input domain. Thus, stress distribution is obtained at each point in the mesh. 
The specific grid point with the maximum mean stress is regarded as the \textit{point of interest},
denoted as P. Our findings revealed that P is infact located at the top right corner of the $x^c$-z$^c$
plane, consistent with the location of the square in Figure~\ref{fig:hs}.
Figure~\ref{fig:kde_S} shows the stress distribution at P. 
%
\begin{figure}[htbp]
\begin{center}
\includegraphics[width=0.42\textwidth]{./Figures/kde_S_mumax}
\end{center}
\caption{}
\label{fig:kde_S}
\end{figure}
%

\subsection{Reliability Prediction}
\label{sub:reliability}




 
