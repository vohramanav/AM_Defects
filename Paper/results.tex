\section{Results}
\label{sec:results}

In this section, we provide relevant details pertaining to the construction of the surrogate for the field of interest, i.e. the
residual stress in the AM part cross-section using the PCAS method in~\ref{sub:surr}. The surrogate is used to map
the process control parameters and the material properties to the residual stress field. The computational efficiency
enabled by the surrogate is exploited to perform a global sensitivity analysis of the inputs in~\ref{sub:gsa}. Finally, the
surrogate is used for reliability prediction for the AM part by estimating the probability of failure based on residual stress
in~\ref{sub:reliability}.

\subsection{Surrogate Model}
\label{sub:surr}

A surrogate model is constructed for the residual stress field at the cross-section of the part 
(x-z plane in Figure~\ref{fig:PartwMesh}) passing through its center. The surrogate maps three sets of
parameters, namely, the process control parameters~($\bm{\theta_p}$), mechanical properties~($\bm{\theta_m}$),
and thermal properties~($\bm{\theta_t}$) to the stress field. The set of process control parameters includes
the beam power~($P$), scan speed~($v$), and the pre-heat temperature~($T_0$). Mechanical properties
include the yield strength~($Y$), the elastic modulus~($E$), and the bulk density~($\rho$). Thermal 
properties include specific heat~($C_p$) and bulk thermal conductivity~($\kappa$). Note that $C_p$ and $\kappa$
are considered to be functions of the local temperature, $T$.



%The first step in the PCAS
%method involves dimension reduction in the output space using principal component analysis. 



\subsection{Global Sensitivity Analysis}
\label{sub:gsa}


\subsection{Reliability Prediction}
\label{sub:reliability}




 
